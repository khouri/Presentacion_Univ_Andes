\section{Consideraciones finales}

\begin{frame}
  	\begin{block}{Mayores contribuciones}
  		\begin{enumerate}
  			\item Una revisión sistemática del área recomendada de actividades en los flujos de trabajo científicos que puede ser la base para el trabajo futuro.
  			\item Se construyó una base de datos relacional de flujos de trabajo científicos con sus respectivas actividades. Esta base estará disponible en su totalidad para su uso por otras obras.
  			\item Se implementaron diferentes técnicas de la literatura relacionada y los resultados de la recomendación de estas técnicas se compararon con los resultados de la solución propuesta.
  			\item Hasta ahora, la investigación de este maestro ha contribuido a la publicación de dos artículos científicos.
  		\end{enumerate}
  		
  	\end{block}
  \end{frame}
	
	\begin{frame}
		\begin{block}{Consideraciones finales}
			Al comparar todas las técnicas, se encontraron ciertos aspectos del conjunto de datos, como el hecho de que las actividades no eran independientes; el problema no es linealmente separable; y que las técnicas de agrupamiento no eran adecuadas para resolver este problema. 
			
			Con la excepción de SVM, los regresores tienen soluciones más precisas que los clasificadores, y también agregan información a los sistemas de recomendación mejoró su precisión.
		\end{block}
	\end{frame}

 \begin{frame}
 	\begin{block}{Trabajos futuros}
 		%No decorrer deste projeto foram identificadas algumas oportunidades de continuidade e evolu\c{c}\~ao do mesmo, s\~ao elas:
 		\begin{enumerate}
 			\item Use otros clasificadores compuestos al recomendar actividades
 			\item Crear nuevas estrategias de recomendación basadas en las redes sociales de investigadores o sus grupos de investigación
 			\item Obtenga información sobre la procedencia \emph{workflows} y agréguela a los sistemas de recomendación;
 		
		\end{enumerate}		
 	\end{block}
 \end{frame}
 
 
 \begin{frame}
 	\begin{block}{Trabajos futuros}
 		%No decorrer deste projeto foram identificadas algumas oportunidades de continuidade e evolu\c{c}\~ao do mesmo, s\~ao elas:
 		\begin{enumerate}
 			\item Utilice actividades de otras áreas de investigación de SGWC y / u otras (además de bioinformática)
 			\item Para estudiar la relación entre la distribución de datos de entrada (actividad), su escasez y la relación que ambos tienen con el aumento o reducción de la precisión de las recomendaciones
 			\item Utilice técnicas de reducción de dimensionalidad para el conjunto de datos de entrada;
 			\item Adaptar el clasificador SVM para considerar las ontologías durante la maximización del margen es óptimo.
 		\end{enumerate}		
 	\end{block}
 \end{frame}